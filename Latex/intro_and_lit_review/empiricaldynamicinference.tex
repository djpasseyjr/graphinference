\documentclass{article}

\usepackage{amsmath}
\usepackage{amssymb}
\usepackage{bm}
\usepackage{hyperref}

\title{Empirical Dynamic Inference: A  methodology for using data to model uncertainty about complex dynamics}
\author{D.J. Passey}

\begin{document}
\maketitle

\section*{Posing the Question}
As we seek to understand the world, identify its forms and comprehend the complex web of connections
that underlies the rich and varying patterns in life, we use models.

Most often, these models are efficient mental constructs that we can call up in our minds in order to simulate,
intervene and make predictions, conscious or unconscious, about the systems in which we live.

However, it is the goal of many philosophers, mathematicians and statisticians to formalize these
normative cognitive processes \cite{woodard2021causation}. By decomposing the mechanisms of cognitive processes into
distinct operations and developing logical constructs and algebras, we gain insight into intelligence
and sometimes develop inferential operations that, in certain context, are more effective than those
of our own minds. Furthermore, the precise logic of these operations and formalisms enables us to translate
them into algorithms that leverage modern compute. These algorithms augment our ability to use observations
to inform decisions and hopefully improve the wellbeing of life on our planet.

Setting aside the issue of how these formalisms are applied, we concern ourselves here with the formalisms
themselves.

\subsection*{Three Classes of Problems}

Before proceeding, we make a loose distinction about three classes of problems.

\begin{enumerate}
    \item \textbf{Class 1:} \textit{Holding the system (or data generating process) constant, how can I
    leverage past observations to predict what will happen next?}
    
    Some examples of this are recommender
    systems, classification problems, time series forecasting, or text completion. Many of the formal
    models used to solve this class of problem fall under the umbrella of machine learning but some, 
    such as probabilistic programming models, do not.

    \item \textbf{Class 2:} \textit{Holding the system constant, how will
    it respond to a previously unobserved intervention?}
    
    This problem occurs when evaluating the effects of proposed
    government policies, estimating the impact of different pharmaceuticals on the population, or understanding how
    to intervene in an ecosystem in order to maintain the survival of a species. The methodologies for studying
    this class of problem include causal inference, system dynamics, structural equation models and generalized regression.

    \item \textbf{Class 3} \textit{How can I change the system, or devise a new system that achieves a
    desired outcome?}
    
    This class of problems is likely the most challenging and may have the greatest potential to benefit
    society. Some examples of this class of problem are product design, invention, questions such as
    ``Can I engineer an ecosystem of organisms that produces food sustainably?'' or
    ``how should we structure society?''. The toolkit for these particular problems remains ill-defined.
    From the vantage point of this author, it appears that these problems are solved via years of intensive
    investigation culminating in expert knowledge of \textit{how things work}. This is often the domain of
    theory---theory that is not always formalized into mathematics, such as Foucault's analysis of the subject and power,
    the nitrogen cycle, or the macro-deleveraging process in economics.

\end{enumerate}

At this point, an attentive reader has likely thought of problem areas that span multiple of the above classes, or tools
that solve multiple problems. It is correct that these problems and their accompanying toolkits are \textit{highly interrelated}.
Advanced tools from a certain class of problem can lead to advances in another class. For example, DragonNet is a neural
network designed to estimate treatment effects from observational data  \cite{shi2019adapting}. 

DragonNet also illustrates how the problems themselves are interrelated. The model's loss function optimizes both
prediction error and treatment assignment error to achieve better causal effect estimates. This illustrates
how prediction, as described in item one above, is related to causal interventions as described in problem two. 

These concepts are related in the sense that a model which can perfectly predict everything, could also accurately
predict treatment effects. However, in practice, algorithms that only minimize prediction error are typically insufficient to 
elucidate causal effect in the presence of confounders. Analogously, models that accurately estimate causal effect 
(such as regressions) have high prediction error. (Note that this does not illustrate some universal
tradeoff beteen causal effect and prediction—it is possible that someone will invent an algorithm that can do both.)

However, this distinction is still useful because it can guide problem solvers to methodologies that are likely to
be helpful as they seek to answer important questions. While it is possible to turn many problems into a machine
learning problems, it may be more profitable to address some questions with causal analysis or PDE simulation.
Even the AI giants like Facebook and Google employ Bayesian techniques and AB testing to evaluate new features.
Similarly engineering firms all over the world use PDEs to simulate products before they are fabricated.

The research presented in this work will focus primarily on problem two and as much as possible attempt to extend results
to problem three. The goal of this work is to begin the development of a formal methodology for leveraging data to effectively
quantify uncertainty during to process of creating, assessing and comparing explanatory dynamic models that seek to accurately
describe real world relationships through time.

    \subsection*{How to Evaluate Success}

        The problem described above is a near universal problem in the sciences, and whether considered directly or indirectly,
        it is confronted in many fields of work. There is an extensive suite of methods for studying time series data and
        drawing conclusions about variable relationships. Yet the kinds of conclusions generated and the assumptions of the
        techniques vary. In addition, the lack of uncertainty quantification makes many of them ill-suited for scientific debate.

        The primary goal of this work to is to increase our ability to explain and understand reality. This goal rests on the
        assumption that there is a logic to the world that can be uncovered and understood. Therefore, we evaluate methodologies
        on the quality of the conclusions that can be drawn from their analyses. This evaluation will include the following:

        \begin{enumerate}
            \item A discussion of the assumptions made by the method and the difficulty of testing the assumptions.
            \item Benchmarking the method on datasets that meet its assumptions.
            \item How well the methodology surrounding the method resolves competing explanations.
            \item An analysis of the ability of the method to make convincing arguments about the structure of the natural world.
        \end{enumerate}

\section*{Discussion of Problem Class 2: Extrapolation}

    \subsection*{Relevance and Importance}

        We'll begin with a few examples of situations where problem 2 is important. For each example, assume that no
        similar intervention has been observed.

        \begin{itemize}
            \item (Ecology) How would coral reef growth be affected if the government increased the length of the fishing season?
            \item (Molecular Biology) How will cell proteins be affected by a growth inhibitory compound \cite{dibernardo2005chemogenomic}? 
            \item (Economics) How would small businesses be affected if we increased the minimum wage in a particular city?
            \item (Social Science) How would increasing the number of black teachers affect future income of black students?
            \item (Systems Biology) What would happen to the interaction between gut microbiome and mood in response to a specific diet?
        \end{itemize}

        It is important to note that machine learning models are not typically trusted to answer these sorts of questions.
        This may be because these questions are strictly concerned with out of distribution data and machine learning
        algorithms have no measure of uncertainty about their predictions. Bayesian models for causal inference estimate
        causal effect and also provide a measure of how well particular causal question could be answered
        by the available data. While it may be the case that on average ML models make more accurate predictions, there is no way of 
        assessing the risk profile of basing policy on the predictions of a particular ML model whereas standard
        statistical techniques lay bare the assumptions and the uncertainty surrounding their predictions.


    \subsection*{Important Concepts}

        \begin{itemize}
            \item \textbf{State space exploration:} An important concept, especially for non-parametric methods is the degree to 
            which the trajectory or set of trajectories explores the space. For the problem of extrapolation, we are concerned with
            an area of the state space that has not been explored, so we are most interested in understanding how much the
            observed data tells us about the effects of our posed intervention.

            \item \textbf{Incorporating known mechanisms:} Often, we have knowledge about the world that is relevant to
            extrapolation. For instance, we know that a change in shark populations will not impact ocean salinity, or that
            a person with a home loan must make payments on that loan or default. The causal inference methodology uses
            causal graphs to incorporate known mechanisms into their analyses, similarly, physics informed ML builds
            physical assumptions into the loss function. It is important to remember that a computational model knows far
            less about the world than the researcher. All it knows is its structure and the data it sees. Incorporating
            mechanisms into a model ensures that when it makes out of distribution predictions, they will at least
            satisfy known mechanisms.

            \item \textbf{Causality and dynamics:} The school of Pearlian causal inference has a strong and precise definition of
            causality and it's practicioners can be eager to cast doubt on alternate formulations-often rightfully so.
            However, the Pearlian school mainly restricts it's analyses to static data and assumes that the direction and
            magnitude of causation is independent of system state (Is this really correct?). Many systems are known to
            exhibit state dependent causal effects and even show the reversal of direction of the effect.

            Additionally, the Pearlian school approaches data with the maxim, ``The causes are not in the data'' illustrating
            that you must approach analyses with strong causal assumptions in order to correctly uncover causal effects.
            While this is a good maxim for static data, time series data contains information about the \textbf{sequence of events}
            which empowers algorithms to make inferences about causes.

            Clearly, in the space of dynamics, we need new definitions of causality to meet our needs. Some helpful info
            \href{https://stats.stackexchange.com/questions/26300/does-causation-imply-correlation#:~:text=The%20strict%20answer%20is%20%22no,does%20not%20necessarily%20imply%20correlation%22.&text=using%20the%20property%20of%20the,correlation%20is%20equal%20to%20zero.}
            {on this stack exchange}.

            \item \textbf{Dimensionality Reduction and Invariant Manifolds:} When studying a high dimensional time series
            it is likely that the space of possible trajectories is much smaller than the size of the hypercube that encloses
            the observed data. Mapping the system to a smaller space can reduce computation and avoid problems
            surrounding high dimensional data. Many techniques exist for doing this, diffusion maps, autoencoders
            \cite{lee2020model, dsilva2018parsimonious}. More investigation is needed to understand how much data they need.
         
        \end{itemize}


\section*{A Survey of Related Work}

    \subsubsection*{Detecting Causality in Complex Ecosystems (Convergent Cross Mapping/Empirical Dynamic Modeling)}

        A 2012 paper that presented a method for identifying causality in non-linear dynamical
        systems \cite{sugihara2012detecting}. It solves three problems with Granger causality: 
        \begin{enumerate}
            \item Granger causality assumes separable variables, that is, by removing a variable,
            $X$ from the data, all information about $X$ is lost and not included in another variable
            \item Granger causality fails to identify weakly coupled variables
            \item Granger causality cannot distinguish interactions from external forces
        \end{enumerate}

        It uses a dynamical systems definition of causality, where two time
        series variables are causally linked if they belong to the same system. (I assume that means 
        that you can't decompose the system into a smaller system that excludes one of the variables.)

        The paper contains interesting ecology datasets such as a sardine anchovy dataset and 
        a paramecium one.

    \subsubsection*{Time Series Analysis (Hamilton)}

        Apparently a seminal text \cite{hamilton1994time}. Focuses mainly on autoregressive models.
        Chapter on Kalman filters. The end of the book discusses concepts like
        cointegration and heteroskedasticity that could be interesting.
        \href{http://mayoral.iae-csic.org/timeseries2021/hamilton.pdf}{Link to pdf.}

    \subsubsection*{Time Series Analysis Handbook}

        Notebooks with code on \href{https://phdinds-aim.github.io/time_series_handbook/Preface/Preface.html}{github}.
        Compiled by PhDs in data science at the Asian Institute of Management in 2020-2023.
        It has convergent cross mapping and empirical dynamic modeling in it with lots of code and datasets.

    \subsubsection*{From Ordinary Differential Equations to Structural Causal Models: The Deterministic Case}
        
        A theoretical work that connects differential equation to Pearlian notions of causality
        and suggests that structural equation models can describe a differential equation \cite{mooij2013ordinary}.

    \subsubsection*{Causal inference for time series}
        
        A nature review paper of the structural causal model methodology \cite{runge2023causal}.
        \href{https://climateinformaticslab.com/wp-content/uploads/2023/06/Runge_Causal_Inference_for_Time_Series_NREE.pdf}
        {Link to PDF.}

    \subsubsection*{Recent developments in empirical dynamic modelling}

        Taken from the abstract: ``Recent extensions of EDM to multivariate
        time series substantially expand the range of applications and mechanistic
        questions that can be addressed, including detecting causal coupling, tracking
        changing interactions in real time, leveraging short time series from information shared
        in coupled variables, modelling dynamically changing stability, scenario exploration, and management
        applications involving optimal control'' \cite{munch2023recent}.

        This makes me wonder if there is a way to incorporate Bayesian techniques
        into this approach.

    \subsubsection*{Ecological Modeling from Time-Series Inference: Insight into Dynamics and Stability of Intestinal Microbiota}

        Uses a generalized Lotka-Voltera to model and assess stability of the intestinal mircobiome \cite{stein2013ecological}.
        
    \subsubsection*{dynGENIE3: dynamical GENIE3 for the inference of gene networks from time series expression data}
        Comapres an ML based method for identifying gene regulatory networks with a number of other methods. Is only
        beaten by Gaussian processes (which is computationally complex) \cite{huynhthu2018dynenie3}. The paper compares 
        13 methods in total on a gene regulatory benchmark dataset. They classify the methods into five categories:
        
        \begin{enumerate}
            \item Tree Ensembles
            \item Mutual information
            \item Dynamic Bayesian networks
            \item ODEs
            \item Non-linear Dynamical Systems
            \item Granger Causality
        \end{enumerate}

        It would be very interesting to learn more about each of these classes of methods.

    \subsubsection*{Modelling individual and cross-cultural variation in the mapping of emotions to speech prosody}
        An excellent example of applied bayesian modeling that investigates the connection between human
        tone and emotion across cultures \cite{vanrijn2023modeling}.

    \subsubsection*{Model reduction of dynamical systems on nonlinear manifolds using deep convolutional autoencoders}
        Uses autoencoders to project dynamical systems onto non-linear manifolds \cite{lee2020model}.
        Has a lot of great theory about dimensionality reduction.

    \subsubsection*{Parsimonious representation of nonlinear dynamical systems through manifold learning: A chemotaxis case study}
        A method for dimensionality reduction that improves an eigenvalue problem with diffusion maps \cite{dsilva2018parsimonious}. Works for
        stochastic differential equations.

    \subsubsection*{Uncertainty quantification for complex systems with very high dimensional response using Grassmann manifold variations}
        Investigates the problem of high uncertainty quantification in complex and high dimensional stochastic physics
        models \cite{giovanis2018uncertainty}. These models have challenges with efficient sampling that can be accelerated with
        dimensionality reduction.

    \subsection*{How to infer gene networks from expression profiles}
        An excellent paper that describes each kind of method for inferring
        connections in gene regulatory networks \cite{bansal2007infer}.

        Excerpt: \textit{To simulate gene expression data and gene regulation in
        the form of a network, we use linear ODEs relating the changes in gene
        transcript concentration to each other and to the external
        perturbations. Linear ODEs can simulate gene networks as directed signed
        graphs with realistic dynamics and generate both steady-state and
        time-series gene expression profiles. Linear ODEs are generic, as any
        non-linear process can be approximated to a linear process, as long as
        the system is not far from equilibrium, whereas non-linear processes are
        all different from each other.}
        
        I'm very surprised that the algorithms are only being tested on linear
        models! Is this actually good enough to make scientific progress? It
        seems like a real non-liner investigation would be very informative in
        understanding all of these methods.

        Simply collecting a bunch of non-linear ode models and running
        simulations, then trying to infer the network seems like a huge
        contribution.
        
        The accuracy of the methods is just terrible too! It's very interesting.

        It seems like their real world dataset results sort of match their
        linear model inference results, so that's good.

\section*{Experimentation}

Each methodology was used to make inferences in three scenarios.
\begin{enumerate}
    \item Simulations of models where the underlying system is completely characterized. 
    (Stochastic differential equations, differential equations with measurement
    error introduced.)

    \begin{enumerate}
        \item Collect and run existing models. Turn the results into a JSON.
        \item Run the models with noise, or make them stochastic somehow
        \item Run the models with perturbations -- like system identification
        from control theory.
    \end{enumerate}

    \item Real world data, where the underlying system is well understood by the academic community.
    (Modeling the evolution of the \href{https://www.sciencedirect.com/science/article/pii/S0045653520316866}{nitrogen}
    cycle or the impact of interest rates on inflation.)
    \item Real world data where the true system is unknown. (While there is no ``ground truth''
    in this scenario, the value here involves assessing how well the methodology provides the
    researcher with confidence about the validity of results.)
\end{enumerate}


\bibliographystyle{plain}
\bibliography{empiricaldynamicinference.bib}

\end{document}