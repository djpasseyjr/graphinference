\documentclass{article}

\usepackage{amsmath}
\usepackage{amssymb}
\usepackage{bm}
\usepackage{hyperref}

\title{Empirical Dynamic Inference: A  methodology for using data to model uncertainty about complex dynamics}
\author{D.J. Passey}

\begin{document}
\maketitle

\section{Posing the Question}
As we seek to understand the world, that is, identify its forms and comprehend the complex web of connections
that underlies the rich and varying patterns of life, we use models.

Most often, these models are efficient mental constructs that we can call up in our minds in order to simulate,
intervene and make predictions about the systems in which we live.

It has been the goal of many philosophers, mathematicians and statisticians to formalize these natural
cognitive processes. The precise logic of these formalisms enabled us to translate these operations into
algorithms that leverage the power of modern compute.

\subsection{Three Classes of Problems}

Before proceeding, we make a loose distinction about three classes of problems.

\begin{enumerate}
    \item There is a class of problems (and accompanying formal models)
    that seek to investigate the question, ``holding the system (or data generating process) constant, how can I
    leverage past observations to predict what will happen next?'' Some examples of this are recommender
    systems, classification problems, time series forecasting, or text completion. Many of the formal
    models used to solve this class of problem fall under the umbrella of machine learning but some, 
    such as probabilistic programming models, do not.

    \item A second class of problems seeks to investigate the question, ``holding the system constant, how will
    it respond to a previously unobserved intervention?''. This problem occurs when evaluating the effects of proposed
    government policies, estimating the impact of different pharmaceuticals on the population, or understanding how
    to intervene in an ecosystem in order to maintain the survival of a species. The methodologies for studying
    this class of problem include causal inference, system dynamics, structural equation models and generalized regression.

    \item The third class of problems, (likely the most challenging) may have the greatest potential to benefit
    society. It investigates the question ``how can I change the system, or devise a new system that achieves a
    desired outcome?''. Some examples of this are product design, invention, questions such as
    ``how should we structure society?'' or ``can I create an ecosystem that produces more food than conventional
    farming''. The toolkit for these particular problems remains ill-defined. From the vantage point of this author,
    it appears that these problems are solved via years of intensive investigation culminating in expert knowledge
    of \textit{how things work}. This is often the domain of theory, theory that is not always formalized into mathematics,
    such as Foucault's analysis of the subject and power, the nitrogen cycle, or the macro-deleveraging process.

\end{enumerate}

At this point, an attentive reader has likely thought of problem areas that span multiple of the above classes, or tools
that solve multiple problems. It is correct that these problems and their accompanying toolkits are \textit{highly interrelated}.
Advanced tools from a certain class of problem can lead to advances in another class. For example, DragonNet is a neural
network designed to estimate treatment effects from observational data  \cite{shi2019adapting}. 

DragonNet also illustrates how the problems themselves are interrelated. The model's loss function optimizes both
prediction error and treatment assignment error to achieve better causal effect estimates. This illustrates
how prediction, as described in item one above, is related to causal interventions as described in problem two. 

These concepts are related in the sense that a model which can perfectly predict everything, could also accurately
predict treatment effects. However, in practice, algorithms that only minimize prediction error are typically insufficient to 
elucidate causal effect in the presence of confounders. Analogously, models that accurately estimate causal effect 
(such as regressions) have high prediction error. (Note that this does not illustrate some universal
tradeoff beteen causal effect and prediction—it is possible that someone will invent an algorithm that can do both.)

However, this distinction is still useful because it can guide problem solvers to methodologies that are likely to
be helpful as they seek to answer important questions. While it is possible to turn many problems into a machine
learning problems, it may be more profitable to address some questions with causal analysis or PDE simulation.
Even the AI giants like Facebook and Google employ Bayesian techniques and AB testing to evaluate new features.
Similarly engineering firms all over the world use PDEs to simulate products before they are fabricated.

The research presented in this work will focus primarily on problem two and as much as possible attempt to extend results
to problem three. The goal of this work is to begin the development of a formal methodology for leveraging data to effectively
quantify uncertainty during to process of creating, assessing and comparing explanatory dynamic models that seek to accurately
describe real world relationships through time. 




\section{A Survey of Related Work}

    \subsection{Detecting Causality in Complex Ecosystems (Convergent Cross Mapping/Empirical Dynamic Modeling)}

        A 2012 paper that presented a method for identifying causality in non-linear dynamical
        systems \cite{sugihara2012detecting}. It solves three problems with Granger causality
        \begin{enumerate}
            \item Granger causality assumes separable variables, that is, by removing a variable,
            $X$ from the data, all information about $X$ is lost and not included in another variable
            \item Granger causality fails to identify weakly coupled variables
            \item Granger causality cannot distinguish interactions from external forces
        \end{enumerate}

        It uses a dynamical systems definition of causality, where two time
        series variables are causally linked if they belong to the same system. (I assume that means 
        that you can't decompose the system into a smaller system that excludes one of the variables.)

        The paper contains interesting ecology datasets such as a sardine anchovy dataset and 
        a paramecium one.

    \subsection{Time Series Analysis (Hamilton)}

        Apparently a seminal text \cite{hamilton1994time}. Focuses mainly on autoregressive models.
        Chapter on Kalman filters. The end of the book discusses concepts like
        cointegration and heteroskedasticity that could be interesting.
        \href{http://mayoral.iae-csic.org/timeseries2021/hamilton.pdf}{Link to pdf.}

    \subsection{Time Series Analysis Handbook}

        Notebooks with code on \href{https://phdinds-aim.github.io/time_series_handbook/Preface/Preface.html}{github}.
        Compiled by PhDs in data science at the Asian Institute of Management in 2020-2023.
        It has convergent cross mapping and empirical dynamic modeling in it with lots of code and datasets.

    \subsection{From Ordinary Differential Equations to Structural Causal Models: The Deterministic Case}
        
        A theoretical work that connects differential equation to Pearlian notions of causality
        and suggests that structural equation models can describe a differential equation \cite{mooij2013ordinary}.

    \subsection{Causal inference for time series}
        
        A nature review paper of the structural causal model methodology \cite{runge2023causal}.
        \href{https://climateinformaticslab.com/wp-content/uploads/2023/06/Runge_Causal_Inference_for_Time_Series_NREE.pdf}
        {Link to PDF.}

    \subsection{Recent developments in empirical dynamic modelling}

        Taken from the abstract: ``Recent extensions of EDM to multivariate
        time series substantially expand the range of applications and mechanistic
        questions that can be addressed, including detecting causal coupling, tracking
        changing interactions in real time, leveraging short time series from information shared
        in coupled variables, modelling dynamically changing stability, scenario exploration, and management
        applications involving optimal control'' \cite{munch2023recent}.

        This makes me wonder if there is a way to incorporate Bayesian techniques
        into this approach.

    \subsection{Ecological Modeling from Time-Series Inference: Insight into Dynamics and Stability of Intestinal Microbiota}

        Uses a generalized Lotka-Voltera to model and assess stability of the intestinal mircobiome \cite{stein2013ecological}.
        
    \subsection{dynGENIE3: dynamical GENIE3 for the inference of gene networks from time series expression data}
        Comapres an ML based method for identifying gene regulatory networks with a number of other methods. Is only
        beaten by Gaussian processes (which is computationally complex) \cite{huynhthu2018dynenie3}.

\section{Experimentation}

Each methodology was used to make inferences in three scenarios.
\begin{enumerate}
    \item Simulations of models where the underlying system is completely characterized. 
    (Stochastic differential equations, differential equations with measurement error introduced.)
    \item Real world data, where the underlying system is well understood by the academic community.
    (Modeling the evolution of the \href{https://www.sciencedirect.com/science/article/pii/S0045653520316866}{nitrogen}
    cycle or the impact of interest rates on inflation.)
    \item Real world data where the true system is unknown. (While there is no ``ground truth''
    in this scenario, the value here involves assessing how well the methodology provides the
    researcher with confidence about the validity of results.)
\end{enumerate}

\bibliographystyle{plain}
\bibliography{empiricaldynamicinference.bib}

\end{document}