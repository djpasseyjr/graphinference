\documentclass{article}
\usepackage{amsmath}

\title{Graph Inference on Timeseries Data}
\author{D.J. Passey}

\begin{document}
\maketitle

\section{Understanding the Limitations of Vector Autoregression}

    In Stock 2001, the authors point out, "The Taylor [rule] is “backward 
    looking” in the sense that the Fed reacts to past information..., 
    and several researchers have argued that Fed behavior is more
    appropriately described by forward looking behavior."
    Though not addressing autoregression directly, this makes an
    interesting point about how autoregression, and most models are
    "backward looking" while real world processes involve
    anticipating the future.
    The authors go on to explain that the Taylor rule was updated
    to respond to the autoregressive prediction of what inflation
    would be in the near future. It is interesting to note that
    while this new model does incoorperate a notion of predicting
    the future, mathematically, it is still completely backward looking.
    Later, the authors mention the following issues: 
    \begin{enumerate}
        \item "The
        standard methods of statistical inference (such as computing 
        standard errors for impulse responses) may give misleading 
        results if some of the variables are highly persistent [8].
        Another limitation is that, without modification, standard 
        VARs miss nonlinearities, conditional heteroskedasticity, and
        drifts or breaks in parameters."
        \item "While useful as a benchmark, small
        VARs of two or three variables are often unstable and thus poor predictors of the future
        (Stock and Watson [1996])."
        \item "However, adding variables to the VAR creates complications, because the
        number of VAR parameters increases as the square of the number of variables: a ninevariable, four-lag VAR has 333 unknown coefficients (including the intercepts).
        Unfortunately, macroeconomic time series data cannot provide reliable estimates of all
        these coefficients without further restrictions."
        \item "a common assumption made in
        structural VARs is that variables like output and inflation
        are sticky and do respond “within the period” to monetary 
        policy shocks. This seems plausible over the period of a 
        single day, but becomes less plausible over a month or
        quarter."
    \end{enumerate}


\end{document}