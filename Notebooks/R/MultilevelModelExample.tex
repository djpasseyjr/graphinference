% Options for packages loaded elsewhere
\PassOptionsToPackage{unicode}{hyperref}
\PassOptionsToPackage{hyphens}{url}
%
\documentclass[
]{article}
\usepackage{amsmath,amssymb}
\usepackage{lmodern}
\usepackage{iftex}
\ifPDFTeX
  \usepackage[T1]{fontenc}
  \usepackage[utf8]{inputenc}
  \usepackage{textcomp} % provide euro and other symbols
\else % if luatex or xetex
  \usepackage{unicode-math}
  \defaultfontfeatures{Scale=MatchLowercase}
  \defaultfontfeatures[\rmfamily]{Ligatures=TeX,Scale=1}
\fi
% Use upquote if available, for straight quotes in verbatim environments
\IfFileExists{upquote.sty}{\usepackage{upquote}}{}
\IfFileExists{microtype.sty}{% use microtype if available
  \usepackage[]{microtype}
  \UseMicrotypeSet[protrusion]{basicmath} % disable protrusion for tt fonts
}{}
\makeatletter
\@ifundefined{KOMAClassName}{% if non-KOMA class
  \IfFileExists{parskip.sty}{%
    \usepackage{parskip}
  }{% else
    \setlength{\parindent}{0pt}
    \setlength{\parskip}{6pt plus 2pt minus 1pt}}
}{% if KOMA class
  \KOMAoptions{parskip=half}}
\makeatother
\usepackage{xcolor}
\usepackage[margin=1in]{geometry}
\usepackage{color}
\usepackage{fancyvrb}
\newcommand{\VerbBar}{|}
\newcommand{\VERB}{\Verb[commandchars=\\\{\}]}
\DefineVerbatimEnvironment{Highlighting}{Verbatim}{commandchars=\\\{\}}
% Add ',fontsize=\small' for more characters per line
\usepackage{framed}
\definecolor{shadecolor}{RGB}{248,248,248}
\newenvironment{Shaded}{\begin{snugshade}}{\end{snugshade}}
\newcommand{\AlertTok}[1]{\textcolor[rgb]{0.94,0.16,0.16}{#1}}
\newcommand{\AnnotationTok}[1]{\textcolor[rgb]{0.56,0.35,0.01}{\textbf{\textit{#1}}}}
\newcommand{\AttributeTok}[1]{\textcolor[rgb]{0.77,0.63,0.00}{#1}}
\newcommand{\BaseNTok}[1]{\textcolor[rgb]{0.00,0.00,0.81}{#1}}
\newcommand{\BuiltInTok}[1]{#1}
\newcommand{\CharTok}[1]{\textcolor[rgb]{0.31,0.60,0.02}{#1}}
\newcommand{\CommentTok}[1]{\textcolor[rgb]{0.56,0.35,0.01}{\textit{#1}}}
\newcommand{\CommentVarTok}[1]{\textcolor[rgb]{0.56,0.35,0.01}{\textbf{\textit{#1}}}}
\newcommand{\ConstantTok}[1]{\textcolor[rgb]{0.00,0.00,0.00}{#1}}
\newcommand{\ControlFlowTok}[1]{\textcolor[rgb]{0.13,0.29,0.53}{\textbf{#1}}}
\newcommand{\DataTypeTok}[1]{\textcolor[rgb]{0.13,0.29,0.53}{#1}}
\newcommand{\DecValTok}[1]{\textcolor[rgb]{0.00,0.00,0.81}{#1}}
\newcommand{\DocumentationTok}[1]{\textcolor[rgb]{0.56,0.35,0.01}{\textbf{\textit{#1}}}}
\newcommand{\ErrorTok}[1]{\textcolor[rgb]{0.64,0.00,0.00}{\textbf{#1}}}
\newcommand{\ExtensionTok}[1]{#1}
\newcommand{\FloatTok}[1]{\textcolor[rgb]{0.00,0.00,0.81}{#1}}
\newcommand{\FunctionTok}[1]{\textcolor[rgb]{0.00,0.00,0.00}{#1}}
\newcommand{\ImportTok}[1]{#1}
\newcommand{\InformationTok}[1]{\textcolor[rgb]{0.56,0.35,0.01}{\textbf{\textit{#1}}}}
\newcommand{\KeywordTok}[1]{\textcolor[rgb]{0.13,0.29,0.53}{\textbf{#1}}}
\newcommand{\NormalTok}[1]{#1}
\newcommand{\OperatorTok}[1]{\textcolor[rgb]{0.81,0.36,0.00}{\textbf{#1}}}
\newcommand{\OtherTok}[1]{\textcolor[rgb]{0.56,0.35,0.01}{#1}}
\newcommand{\PreprocessorTok}[1]{\textcolor[rgb]{0.56,0.35,0.01}{\textit{#1}}}
\newcommand{\RegionMarkerTok}[1]{#1}
\newcommand{\SpecialCharTok}[1]{\textcolor[rgb]{0.00,0.00,0.00}{#1}}
\newcommand{\SpecialStringTok}[1]{\textcolor[rgb]{0.31,0.60,0.02}{#1}}
\newcommand{\StringTok}[1]{\textcolor[rgb]{0.31,0.60,0.02}{#1}}
\newcommand{\VariableTok}[1]{\textcolor[rgb]{0.00,0.00,0.00}{#1}}
\newcommand{\VerbatimStringTok}[1]{\textcolor[rgb]{0.31,0.60,0.02}{#1}}
\newcommand{\WarningTok}[1]{\textcolor[rgb]{0.56,0.35,0.01}{\textbf{\textit{#1}}}}
\usepackage{graphicx}
\makeatletter
\def\maxwidth{\ifdim\Gin@nat@width>\linewidth\linewidth\else\Gin@nat@width\fi}
\def\maxheight{\ifdim\Gin@nat@height>\textheight\textheight\else\Gin@nat@height\fi}
\makeatother
% Scale images if necessary, so that they will not overflow the page
% margins by default, and it is still possible to overwrite the defaults
% using explicit options in \includegraphics[width, height, ...]{}
\setkeys{Gin}{width=\maxwidth,height=\maxheight,keepaspectratio}
% Set default figure placement to htbp
\makeatletter
\def\fps@figure{htbp}
\makeatother
\setlength{\emergencystretch}{3em} % prevent overfull lines
\providecommand{\tightlist}{%
  \setlength{\itemsep}{0pt}\setlength{\parskip}{0pt}}
\setcounter{secnumdepth}{-\maxdimen} % remove section numbering
\renewcommand{\familydefault}{\sfdefault}
\ifLuaTeX
  \usepackage{selnolig}  % disable illegal ligatures
\fi
\IfFileExists{bookmark.sty}{\usepackage{bookmark}}{\usepackage{hyperref}}
\IfFileExists{xurl.sty}{\usepackage{xurl}}{} % add URL line breaks if available
\urlstyle{same} % disable monospaced font for URLs
\hypersetup{
  pdftitle={Exploring and Simulating With a Multilevel Model},
  pdfauthor={DJ Passey},
  hidelinks,
  pdfcreator={LaTeX via pandoc}}

\title{Exploring and Simulating With a Multilevel Model}
\author{DJ Passey}
\date{Jan 4, 2023}

\begin{document}
\maketitle

\begin{Shaded}
\begin{Highlighting}[]
\FunctionTok{library}\NormalTok{(renv)}
\end{Highlighting}
\end{Shaded}

\begin{verbatim}
## 
## Attaching package: 'renv'
\end{verbatim}

\begin{verbatim}
## The following objects are masked from 'package:stats':
## 
##     embed, update
\end{verbatim}

\begin{verbatim}
## The following objects are masked from 'package:utils':
## 
##     history, upgrade
\end{verbatim}

\begin{verbatim}
## The following objects are masked from 'package:base':
## 
##     autoload, load, remove
\end{verbatim}

\begin{Shaded}
\begin{Highlighting}[]
\FunctionTok{library}\NormalTok{(here)}
\end{Highlighting}
\end{Shaded}

\begin{verbatim}
## here() starts at /Users/djpassey/Code/graphinference
\end{verbatim}

\begin{Shaded}
\begin{Highlighting}[]
\CommentTok{\# Activate R virtual environment}
\CommentTok{\# The \textasciigrave{}here()\textasciigrave{} function should locate the top level}
\CommentTok{\# directory of the enclosing git repository called \textasciigrave{}graphinference\textasciigrave{}}
\NormalTok{proj\_dir }\OtherTok{=} \FunctionTok{here}\NormalTok{()}
\FunctionTok{print}\NormalTok{(}\FunctionTok{cat}\NormalTok{(}\StringTok{"Double check that project directory}\SpecialCharTok{\textbackslash{}n}\StringTok{"}\NormalTok{,}
          \StringTok{"is named graphinference:}\SpecialCharTok{\textbackslash{}n}\StringTok{"}\NormalTok{, }
          \StringTok{"proj\_dir ="}\NormalTok{, proj\_dir, }\StringTok{"}\SpecialCharTok{\textbackslash{}n}\StringTok{"}\NormalTok{))}
\end{Highlighting}
\end{Shaded}

\begin{verbatim}
## Double check that project directory
##  is named graphinference:
##  proj_dir = /Users/djpassey/Code/graphinference 
## NULL
\end{verbatim}

\begin{Shaded}
\begin{Highlighting}[]
\NormalTok{renv}\SpecialCharTok{::}\FunctionTok{activate}\NormalTok{(proj\_dir)}

\FunctionTok{library}\NormalTok{(lme4)}
\end{Highlighting}
\end{Shaded}

\begin{verbatim}
## Loading required package: Matrix
\end{verbatim}

\hypertarget{mixed-effect-vs-multilevel-model}{%
\section{Mixed Effect vs Multilevel
Model}\label{mixed-effect-vs-multilevel-model}}

\hypertarget{mixed-effect-model}{%
\subsubsection{Mixed Effect Model}\label{mixed-effect-model}}

A mixed effect model can be written in the form: \[
Y =\mu + X \beta + Z \alpha + \epsilon
\] Here, \(\beta\) represents the fixed effects, and \(\alpha\)
represents the random effects. The matrix \(X\) contains all the factors
from the data, either continuous or one hot encoded categorical values.

The matrix \(Z\) includes the random effect factors. If no random slopes
are to be estimated, then Z will be a zero-one matrix. In this case, the
\(\alpha\) values are effectively intercepts corresponding to class
inclusion.

However, when random slopes are included in the model we model class
inclusion as having an impact on the slope of the individuals response
to a given variable. For example, if we are measuring blood pressure and
predicting mood, we might want to model each person in the study as
having a random intercept (some are happier than others) and also having
different relationships between blood pressure and mood. For some,
higher blood pressure means they are excited and happy, for others,
higher blood pressure means they are anxious.

In this case, a row of the \(Z\) matrix would have \(n\) entries where
only one of them, entry \(i\), is equal to one. This is the one hot
encoding of person ID and corresponds to the random intercept. The next
\(n\) entries would correspond to blood pressure, one hot encoded to
align with person ID. That is, at the same relative location in the
second half of the row, entry \(n + i\), the \(Z\) matrix would have the
person's blood pressure and every other entry between \(n\) and \(2n\)
would be zero.

Thus \(\alpha_i\) is the random intercept of mood for person \(i\) and
\(\alpha_{n+i}\) is the random slope that models how the blood pressure
of person \(i\) affects their mood.

However, the model does not estimate the random effects. It only
estimates the \emph{variance} of the random effects.

Therefore, to make a prediction, about person \(i\) you would need to
draw a random intercept and slope from normal distributions with zero
mean and the estimated variances of the random intercept and slope
respectively. Then you could pass data to your model.

What is weird about this is that mixed effect models predict differently
every time. In order to capture person specific information, you would
probably want to fit a new regression without random effects just to
that specific person.

Great \href{https://www.youtube.com/watch?v=n_lz5I4POqk\&t=219s}{video}
about this here.

\hypertarget{multilevel-model}{%
\subsubsection{Multilevel model}\label{multilevel-model}}

A multilevel model is a specific kind of mixed effect model. It is
usually written as two ``levels''. The example equations given here
represent the least parsimonious multilevel model, given the number of
covariates considered. It is possible to specify multilevel models that
omit some of the following terms.

\textbf{Level one:} \[
y_{ij} = \beta_{0j} + \beta_{1j} \, x_{1ij} + \beta_{2j}\, x_{2ij} + e_{ij}
\] Here \(y_ij\) represents the dependent variable. The index \(j\)
denotes the group and ranges from \(1, 2, ..., g\) and the variable
\(i\) denotes the within-group index and ranges from \(1, 2, ..., N_g\).
Each group can have a different number of observed values of \(y\), thus
\(N_j\), the total number of observations varies based on group.

For example, a modeler might try to predict income, grouping by race. If
there are five races \(g = 5\). If the first race, \(j=1\), is Native
American and the second race, \(j=2\) is Asian and the data includes 300
Native American income values and 100 Asian income values, then
\(N_1 = 300\) and \(N_2=100\).

The next value, \(\beta_{0j}\) is an intercept that varies depending on
the group. Continuing with our previous example, \(\beta_{02}\) would be
the intercept for Asian income.

The coefficient \(\beta_{1j}\) is a coefficient that varies based on
group. It represent the effect of the first factor in the model
\(x_{1ij}\) on income. If the first factor is education, then the group
indexing of \(\beta_{1j}\) allows the effect of education on income to
vary among the groups.

The first factor is represented by \(x_{1ij}\). The first index is 1 and
denotes that this is the first factor in the model. The index \(j\)
denotes the group membership of this observation and \(i\) denotes the
index of the observation as it relates to the total number of
observations within the group \(j\).

The next term \(\beta_{2j} x_{2ij}\) has the same interpretation as the
term above, except relating to the second factor in the model.
Continuing with our example, \(x_{2ij}\) might represent age.

We assume \(e_{ij} \sim \text{N}(0, \sigma_e)\).

\textbf{Level Two: } In a multilevel model, we can describe each of the
\(\beta\) values from the level one equation with another equation.

\[
\beta_{0j} = \gamma_{00} + \gamma_{01}w_{1j} + \gamma_{02}w_{2j} + u_{0j} \\
\beta_{1j} = \gamma_{10} + \gamma_{11}w_{1j} + \gamma_{12}w_{2j} + u_{1j} \\
\beta_{2j} = \gamma_{20} + \gamma_{21}w_{1j} + \gamma_{22}w_{2j} + u_{2j} \\
\] In each of these equations, we describe \(\beta_{kj}\) a linear
combination of an intercept, group specific covariates, \(w1j\) and
\(w2j\), and an error term, \(u_{kj}\).

It is important that the group specific covariates take on only one
value per group. In our example where the data is grouped by race,
\(w1j\) might represent the percent of the United States that identifies
as that race. Then, \(w_{11}\) be the percent of the U.S. that
identifies as Native American because \(j=1\) is the group index for
Native American. Notice that to know the value of \(w_{1j}\) all we need
is a specific group index \(j\). As another example is \(w_{2j}\) might
equal the number of millionaires who identify as that race. Once again,
the special covariates \(w_{1j}\) and \(w_{2j}\) are uniform accross the
\emph{entire} group and do not vary within the group.

As a counter-example, \(w_{1j}\) and \(w_{2j}\) cannot correspond to
variables like age, because this covariate is not the same for everyone
of a particular race. To see that this aligns with the notation, let's
imagine that \(w_{1j}\) did correspond to age. If this were true we
would need an additional index, e.g.~\(w_{1ij}\) to denote variation in
the value of \(w_{1j}\) that cannot be explained by group membership
alone.

This requirement on the \(w\)s comes from the assumption that some
covariates are ``nested''. (But if a covariate is not nested, it can
still be included in the model as an additional covariate
e.g.~\(x_{3ij}\).)

The variable \(u_{kj}\)s are modeled as correlated mean zero gaussian
random variables.

\hypertarget{interactions}{%
\subsubsection{Interactions}\label{interactions}}

For the second level two equation, we recall that in our example,
\(beta_{1j}\) models the effect of education on income. Since \[
\beta_{1j} x_{1ij}=\big( \gamma_{10} + \gamma_{11}w_{1j} + \gamma_{12}w_{2j} + u_{1j} \big) x_{1ij} \\
\beta_{1j} x_{1ij}= \gamma_{10}\, x_{1ij} + \gamma_{11}w_{1j}\, x_{1ij} + \gamma_{12}w_{2j} \, x_{1ij} + u_{1j} \, x_{1ij}
\] each sub-term of \(\beta_{1j}\) captures a different way that group,
and group covariates relate to the dependent variable.

In our example, the first term \(\gamma_{10}x_{1ij}\) is a fixed effect
that models the overall linear relationship between education and income
without taking race into account.

The next two terms,
\(\gamma_{11}w_{1j}\, x_{1ij} + \gamma_{12}w_{2j} \, x_{1ij}\) are
interaction terms that represent the interaction between \(x_{1ij}\) and
the group specific covariates \(w_{1j}\) and \(w_{2j}\). In our example
these represent the interaction between education and the percent of the
population that identifies as race \(j\) and education and the number of
CEOs that identify as race \(j\). (Our example breaks dow here as these
interaction terms seem sort of silly to model.)

The last term \(u_{1j} \, x_{1ij}\) represents a random slope that
describes the manner in which the relationship between education and
income changes for each race. We model the standard deviation and
correlation of \(u\)s just like we do for normal random effects.

\hypertarget{reduced-equation}{%
\subsubsection{Reduced Equation}\label{reduced-equation}}

The reduced equation is when we replace each \(\beta\) with its level 2
equation. This gives:

\$\$ y\_\{ij\} = \gamma\emph{\{00\} + \gamma}\{01\}w\_\{1j\} +
\gamma\emph{\{02\}w}\{2j\} + u\_\{0j\} \textbackslash{}

\qquad + \gamma\emph{\{10\}, x}\{1ij\} + \gamma\emph{\{11\}w}\{1j\} ,
x\_\{1ij\} + \gamma\emph{\{12\}w}\{2j\} , x\_\{1ij\} + u\_\{1j\} ,
x\_\{1ij\} \textbackslash{}

\qquad + \gamma\emph{\{20\}, x}\{2ij\} + \gamma\emph{\{21\}w}\{1j\} ,
x\_\{2ij\} + \gamma\emph{\{22\}w}\{2j\} , x\_\{2ij\} + u\_\{2j\} ,
x\_\{2ij\} \$\$ At this point, we have reduced the multilevel model to a
mixed effect model with interaction terms on nested group covariates. We
can reorganize this equation and group the terms by type:

\$\$ y\_\{ij\} = \gamma\_\{00\}
\text{ (fixed effect intercept)}\textbackslash{}

\qquad + \gamma\emph{\{10\} , x}\{1ij\} + \gamma\emph{\{20\}, x}\{2ij\}
\text{ (fixed effects of covariates)} \textbackslash{}

\begin{itemize}
\item
  \gamma\emph{\{01\}w}\{1j\} + \gamma\emph{\{02\}w}\{2j\}
  \text{ (fixed effects of group covariates)}\textbackslash{}
\item
  \gamma\emph{\{11\}w}\{1j\} , x\_\{1ij\} + \gamma\emph{\{12\}w}\{2j\} ,
  x\_\{1ij\} + \gamma\emph{\{21\}w}\{1j\} , x\_\{2ij\} +
  \gamma\emph{\{22\}w}\{2j\} , x\_\{2ij\}
  \text{ (interaction terms)}\textbackslash{}
\item
  u\_\{0j\} + u\_\{2j\} , x\_\{2ij\} + u\_\{1j\} , x\_\{1ij\}
  \text{( random intercepts and slopes)}\textbackslash{} \$\$
\end{itemize}

\hypertarget{dataset-insteval}{%
\section{\texorpdfstring{Dataset:
\texttt{InstEval}}{Dataset: InstEval}}\label{dataset-insteval}}

InstEval is a dataset of how students rate lectures. The dataset was
created with the goal of identifying the most liked instructor. The
columns are:

\begin{itemize}
\item
  \texttt{s} a factor with levels 1:2972 denoting individual students.
\item
  \texttt{d} a factor with 1128 levels from 1:2160, denoting individual
  professors or lecturers.
\item
  \texttt{studage} an ordered factor with levels 2 \textless{} 4
  \textless{} 6 \textless{} 8, denoting student's ``age'' measured in
  the semester number the student has been enrolled.
\item
  \texttt{lectage} an ordered factor with 6 levels, 1 \textless{} 2
  \textless{} \ldots{} \textless{} 6, measuring how many semesters back
  the lecture rated had taken place.
\item
  \texttt{service} a binary factor with levels 0 and 1; a lecture is a
  ``service'', if held for a different department than the lecturer's
  main one.
\item
  \texttt{dept} a factor with 14 levels from 1:15, using a random code
  for the department of the lecture.
\item
  \texttt{y} a numeric vector of ratings of lectures by the students,
  using the discrete scale 1:5, with meanings of `poor' to `very good'.
\end{itemize}

Each observation is one student's rating for a specific lecture (of one
lecturer, during one semester in the past).

If our goal is to find the best liked lecturer, we are interested in the
relative sizes of the effect of \texttt{d} on \texttt{y}. The best liked
lecturer should have the biggest positive effect on \texttt{y}.

However, we would like to include in the regression the fact that some
students may give mainly positive ratings while others may give more
negative ones. Similarly, students who are older may give different
types of ratings. We can control for this by including a random
intercept for each student and a random intercept for student age.

Similarly, some departments may get higher ratings because their
subjects are well liked. We can control for this so that the effect of
the professor is department agnostic by including a random intercept for
department.

There is a column for service denoting if a professor lectured in their
main department or a different one. It is possible that this column has
interactions with department and professor, that is, for some
departments, service may improve your rating, and for others, service
may decrease your rating. Similarly, there may be professors who do
better outside their department, and other professors who don't.

My inclanation is to control for this with a random intercept on
\texttt{service:d}

Check that instructor (\texttt{d} column) is nested in \texttt{dept}. In
other words, each instructor belongs to a single department.

\begin{Shaded}
\begin{Highlighting}[]
\FunctionTok{all}\NormalTok{(}\FunctionTok{rowSums}\NormalTok{(}\FunctionTok{xtabs}\NormalTok{(}\SpecialCharTok{\textasciitilde{}}\NormalTok{ d }\SpecialCharTok{+}\NormalTok{ dept, InstEval) }\SpecialCharTok{!=}\NormalTok{ 0L ) }\SpecialCharTok{==} \DecValTok{1}\NormalTok{)}
\end{Highlighting}
\end{Shaded}

\begin{verbatim}
## [1] TRUE
\end{verbatim}

Check that each student is associated with exactly one studage.
(Otherwise we would have ratings from students across multiple years.)

\begin{Shaded}
\begin{Highlighting}[]
\FunctionTok{all}\NormalTok{(}\FunctionTok{rowSums}\NormalTok{(}\FunctionTok{xtabs}\NormalTok{(}\SpecialCharTok{\textasciitilde{}}\NormalTok{ s }\SpecialCharTok{+}\NormalTok{ studage, InstEval) }\SpecialCharTok{!=}\NormalTok{ 0L ) }\SpecialCharTok{==} \DecValTok{1}\NormalTok{)}
\end{Highlighting}
\end{Shaded}

\begin{verbatim}
## [1] TRUE
\end{verbatim}

\hypertarget{regressions}{%
\subsection{Regressions}\label{regressions}}

\begin{Shaded}
\begin{Highlighting}[]
\NormalTok{instr\_eval1 }\OtherTok{\textless{}{-}} \FunctionTok{lm}\NormalTok{(y }\SpecialCharTok{\textasciitilde{}}\NormalTok{ d, }\AttributeTok{data=}\NormalTok{InstEval)}
\FunctionTok{summary}\NormalTok{(instr\_eval1)}
\end{Highlighting}
\end{Shaded}

\begin{verbatim}
## 
## Call:
## lm(formula = y ~ d, data = InstEval)
## 
## Residuals:
##     Min      1Q  Median      3Q     Max 
## -3.3636 -0.9298  0.0588  0.9813  3.3333 
## 
## Coefficients:
##               Estimate Std. Error t value Pr(>|t|)    
## (Intercept)  3.727e+00  3.685e-01  10.113  < 2e-16 ***
## d6          -9.531e-01  4.290e-01  -2.222 0.026305 *  
## d7           3.333e-01  4.256e-01   0.783 0.433468    
## d8          -1.134e+00  4.014e-01  -2.825 0.004730 ** 
## d12         -1.606e-01  4.309e-01  -0.373 0.709324    
## d13         -3.497e-02  4.173e-01  -0.084 0.933225    
## ...
## ---
## Signif. codes:  0 '***' 0.001 '**' 0.01 '*' 0.05 '.' 0.1 ' ' 1
## 
## Residual standard error: 1.222 on 72293 degrees of freedom
## Multiple R-squared:  0.1725, Adjusted R-squared:  0.1596 
## F-statistic: 13.37 on 1127 and 72293 DF,  p-value: < 2.2e-16
\end{verbatim}

\begin{Shaded}
\begin{Highlighting}[]
\NormalTok{instr\_eval2 }\OtherTok{\textless{}{-}} \FunctionTok{lm}\NormalTok{(y }\SpecialCharTok{\textasciitilde{}}\NormalTok{ service }\SpecialCharTok{+}\NormalTok{ dept }\SpecialCharTok{+}\NormalTok{ d, }\AttributeTok{data=}\NormalTok{InstEval)}
\FunctionTok{summary}\NormalTok{(instr\_eval2)}
\end{Highlighting}
\end{Shaded}

\begin{verbatim}
## 
## Call:
## lm(formula = y ~ service + dept + d, data = InstEval)
## 
## Residuals:
##     Min      1Q  Median      3Q     Max 
## -3.4308 -0.9198  0.0595  0.9708  3.2984 
## 
## Coefficients: (13 not defined because of singularities)
##               Estimate Std. Error t value Pr(>|t|)    
## (Intercept)  3.7272727  0.3684220  10.117  < 2e-16 ***
## service1    -0.0973856  0.0136723  -7.123 1.07e-12 ***
## dept5       -0.5809547  0.4850535  -1.198 0.231033    
## dept10       0.0505051  0.4209623   0.120 0.904503    
## dept12      -0.7714910  0.4396144  -1.755 0.079276 .  
## dept6       -1.5734266  0.4395008  -3.580 0.000344 ***
## dept7       -1.3272727  0.4850500  -2.736 0.006214 ** 
## dept4       -0.1939394  0.4850500  -0.400 0.689280    
## dept8       -0.1071599  0.4121351  -0.260 0.794856    
## dept9       -0.0877652  0.3809535  -0.230 0.817795    
## dept14      -1.2209398  0.3836084  -3.183 0.001459 ** 
## dept1       -0.2272727  0.5338941  -0.426 0.670336    
## dept3       -0.4965035  0.5005869  -0.992 0.321278    
## dept11       0.0766488  0.3999897   0.192 0.848035    
## dept2       -0.7936066  0.3869406  -2.051 0.040273 *  
## d6          -0.1784467  0.3250867  -0.549 0.583062    
## d7           0.3540701  0.2633004   1.345 0.178714    
## d8          -0.3510072  0.2877547  -1.220 0.222539    
## d12          1.1414884  0.2471467   4.619 3.87e-06 ***
## d13         -0.0754818  0.2824192  -0.267 0.789263    
## ...
## ---
## Signif. codes:  0 '***' 0.001 '**' 0.01 '*' 0.05 '.' 0.1 ' ' 1
## 
## Residual standard error: 1.222 on 72292 degrees of freedom
## Multiple R-squared:  0.173,  Adjusted R-squared:  0.1601 
## F-statistic: 13.41 on 1128 and 72292 DF,  p-value: < 2.2e-16
\end{verbatim}

\begin{Shaded}
\begin{Highlighting}[]
\NormalTok{instr\_eval3 }\OtherTok{\textless{}{-}} \FunctionTok{lm}\NormalTok{(y }\SpecialCharTok{\textasciitilde{}}\NormalTok{ service }\SpecialCharTok{+}\NormalTok{ dept }\SpecialCharTok{+}\NormalTok{ d }\SpecialCharTok{+}\NormalTok{ service}\SpecialCharTok{:}\NormalTok{d, }\AttributeTok{data=}\NormalTok{InstEval)}
\FunctionTok{summary}\NormalTok{(instr\_eval3)}
\end{Highlighting}
\end{Shaded}

\begin{verbatim}
## 
## Call:
## lm(formula = y ~ service + dept + d + service:d, data = InstEval)
## 
## Residuals:
##     Min      1Q  Median      3Q     Max 
## -3.4545 -0.9032  0.0556  0.9592  3.3846 
## 
## Coefficients: (479 not defined because of singularities)
##                  Estimate Std. Error t value Pr(>|t|)    
## (Intercept)     3.727e+00  3.663e-01  10.175  < 2e-16 ***
## service1       -1.238e+00  4.748e-01  -2.606 0.009150 ** 
## dept5          -4.196e-01  4.977e-01  -0.843 0.399217    
## dept10          5.051e-02  4.185e-01   0.121 0.903953    
## dept12         -1.442e+00  5.874e-01  -2.454 0.014123 *  
## dept6          -1.573e+00  4.370e-01  -3.601 0.000318 ***
## dept7          -1.327e+00  4.823e-01  -2.752 0.005921 ** 
## dept4          -1.939e-01  4.823e-01  -0.402 0.687577    
## dept8           1.033e+00  6.270e-01   1.648 0.099453 .  
## dept9          -5.622e-02  3.793e-01  -0.148 0.882177    
## dept14         -1.760e-02  4.264e-01  -0.041 0.967082    
## dept1          -2.273e-01  5.308e-01  -0.428 0.668542    
## dept3          -4.965e-01  4.977e-01  -0.998 0.318486    
## dept11          2.727e-01  9.339e-01   0.292 0.770262    
## dept2           2.727e-01  5.874e-01   0.464 0.642432    
## d6              4.143e-01  5.100e-01   0.812 0.416562    
## d7              1.298e+00  1.004e+00   1.293 0.196029    
## d8              1.951e-01  4.891e-01   0.399 0.690049    
## d12            -1.310e+00  5.855e-01  -2.237 0.025297 *  
## d13            -3.492e-02  2.884e-01  -0.121 0.903622    
## ...
## ---
## Signif. codes:  0 '***' 0.001 '**' 0.01 '*' 0.05 '.' 0.1 ' ' 1
## 
## Residual standard error: 1.215 on 71631 degrees of freedom
## Multiple R-squared:   0.19,  Adjusted R-squared:  0.1698 
## F-statistic: 9.392 on 1789 and 71631 DF,  p-value: < 2.2e-16
\end{verbatim}

\begin{Shaded}
\begin{Highlighting}[]
\FunctionTok{anova}\NormalTok{(instr\_eval2, instr\_eval3)}
\end{Highlighting}
\end{Shaded}

\begin{verbatim}
## Analysis of Variance Table
## 
## Model 1: y ~ service + dept + d
## Model 2: y ~ service + dept + d + service:d
##   Res.Df    RSS  Df Sum of Sq      F    Pr(>F)    
## 1  72292 107938                                   
## 2  71631 105724 661    2213.8 2.2692 < 2.2e-16 ***
## ---
## Signif. codes:  0 '***' 0.001 '**' 0.01 '*' 0.05 '.' 0.1 ' ' 1
\end{verbatim}

\hypertarget{mixed-effect-models}{%
\subsection{Mixed Effect Models}\label{mixed-effect-models}}

\begin{Shaded}
\begin{Highlighting}[]
\NormalTok{me\_instr\_eval1 }\OtherTok{\textless{}{-}} \FunctionTok{lmer}\NormalTok{(y }\SpecialCharTok{\textasciitilde{}}\NormalTok{ service }\SpecialCharTok{+}\NormalTok{ dept }\SpecialCharTok{+}\NormalTok{ d }\SpecialCharTok{+}\NormalTok{ d}\SpecialCharTok{:}\NormalTok{service }\SpecialCharTok{+}\NormalTok{ (}\DecValTok{1} \SpecialCharTok{|}\NormalTok{ studage }\SpecialCharTok{/}\NormalTok{ s), }\AttributeTok{data=}\NormalTok{InstEval)}
\end{Highlighting}
\end{Shaded}

\begin{verbatim}
## fixed-effect model matrix is rank deficient so dropping 479 columns / coefficients
\end{verbatim}

\begin{Shaded}
\begin{Highlighting}[]
\FunctionTok{summary}\NormalTok{(me\_instr\_eval1)}
\end{Highlighting}
\end{Shaded}

\begin{verbatim}
## Linear mixed model fit by REML ['lmerMod']
## Formula: y ~ service + dept + d + d:service + (1 | studage/s)
##    Data: InstEval
## 
## REML criterion at convergence: 234042
## 
## Scaled residuals: 
##      Min       1Q   Median       3Q      Max 
## -2.97146 -0.73061  0.03439  0.75432  3.04260 
## 
## Random effects:
##  Groups    Name        Variance  Std.Dev.
##  s:studage (Intercept) 0.1042016 0.32280 
##  studage   (Intercept) 0.0001142 0.01069 
##  Residual              1.3710352 1.17091 
## Number of obs: 73421, groups:  s:studage, 2972; studage, 4
## 
## Fixed effects:
##                  Estimate Std. Error t value
## (Intercept)     3.8136279  0.3570932  10.680
## service1       -1.0872627  0.4648478  -2.339
## dept5          -0.3588668  0.4856483  -0.739
## dept10         -0.0461133  0.4083123  -0.113
## dept12         -1.6111457  0.5720680  -2.816
## dept6          -1.5423850  0.4262494  -3.619
## dept7          -1.2465021  0.4699869  -2.652
## dept4          -0.1863719  0.4700122  -0.397
## dept8           0.8015823  0.6132148   1.307
## dept9          -0.1356342  0.3700464  -0.367
## dept14         -0.0698464  0.4162224  -0.168
## dept1          -0.3740825  0.5188163  -0.721
## dept3          -0.6951642  0.4867751  -1.428
## dept11          0.5016004  0.9113838   0.550
## dept2           0.0640467  0.5739985   0.112
## d6              0.4779126  0.4954382   0.965
## d7              0.7148472  0.9808910   0.729
## d8              0.3535770  0.4761726   0.743
## d12            -1.2978312  0.5725438  -2.267
## d13            -0.1556240  0.2820550  -0.552
## ...
\end{verbatim}

\begin{verbatim}
## 
## Correlation matrix not shown by default, as p = 1790 > 12.
## Use print(x, correlation=TRUE)  or
##     vcov(x)        if you need it
\end{verbatim}

\begin{verbatim}
## fit warnings:
## fixed-effect model matrix is rank deficient so dropping 479 columns / coefficients
\end{verbatim}

\begin{Shaded}
\begin{Highlighting}[]
\FunctionTok{anova}\NormalTok{(me\_instr\_eval1)}
\end{Highlighting}
\end{Shaded}

\begin{verbatim}
## Analysis of Variance Table
##           npar  Sum Sq Mean Sq  F value
## service      1   258.8 258.832 188.7859
## dept        13   855.8  65.828  48.0135
## d         1114 21049.8  18.896  13.7820
## service:d  661  2043.8   3.092   2.2552
\end{verbatim}

\begin{Shaded}
\begin{Highlighting}[]
\NormalTok{me\_instr\_eval2 }\OtherTok{\textless{}{-}} \FunctionTok{lmer}\NormalTok{(y }\SpecialCharTok{\textasciitilde{}}\NormalTok{ service }\SpecialCharTok{+}\NormalTok{ dept }\SpecialCharTok{+}\NormalTok{ d }\SpecialCharTok{+}\NormalTok{ d}\SpecialCharTok{:}\NormalTok{service }\SpecialCharTok{+}\NormalTok{ (}\DecValTok{1} \SpecialCharTok{|}\NormalTok{ s), }\AttributeTok{data=}\NormalTok{InstEval)}
\end{Highlighting}
\end{Shaded}

\begin{verbatim}
## fixed-effect model matrix is rank deficient so dropping 479 columns / coefficients
\end{verbatim}

\begin{Shaded}
\begin{Highlighting}[]
\FunctionTok{summary}\NormalTok{(me\_instr\_eval2)}
\end{Highlighting}
\end{Shaded}

\begin{verbatim}
## Linear mixed model fit by REML ['lmerMod']
## Formula: y ~ service + dept + d + d:service + (1 | s)
##    Data: InstEval
## 
## REML criterion at convergence: 234042.3
## 
## Scaled residuals: 
##      Min       1Q   Median       3Q      Max 
## -2.97357 -0.73059  0.03469  0.75437  3.04387 
## 
## Random effects:
##  Groups   Name        Variance Std.Dev.
##  s        (Intercept) 0.1043   0.323   
##  Residual             1.3710   1.171   
## Number of obs: 73421, groups:  s, 2972
## 
## Fixed effects:
##                  Estimate Std. Error t value
## (Intercept)     3.8118440  0.3570497  10.676
## service1       -1.0858679  0.4648452  -2.336
## dept5          -0.3597357  0.4856457  -0.741
## dept10         -0.0450186  0.4083066  -0.110
## dept12         -1.6098739  0.5720656  -2.814
## dept6          -1.5402580  0.4262409  -3.614
## dept7          -1.2435485  0.4699753  -2.646
## dept4          -0.1857524  0.4700105  -0.395
## dept8           0.8013665  0.6132142   1.307
## dept9          -0.1352919  0.3700461  -0.366
## dept14         -0.0690918  0.4162202  -0.166
## dept1          -0.3717585  0.5188115  -0.717
## dept3          -0.6924945  0.4867664  -1.423
## dept11          0.5028416  0.9113822   0.552
## dept2           0.0641471  0.5739978   0.112
## d6              0.4772810  0.4954355   0.963
## d7              0.7136243  0.9808891   0.728
## d8              0.3531276  0.4761650   0.742
## d12            -1.2964509  0.5725443  -2.264
## d13            -0.1551402  0.2820446  -0.550
## ...
\end{verbatim}

\begin{verbatim}
## 
## Correlation matrix not shown by default, as p = 1790 > 12.
## Use print(x, correlation=TRUE)  or
##     vcov(x)        if you need it
\end{verbatim}

\begin{verbatim}
## fit warnings:
## fixed-effect model matrix is rank deficient so dropping 479 columns / coefficients
\end{verbatim}

\begin{Shaded}
\begin{Highlighting}[]
\FunctionTok{anova}\NormalTok{(me\_instr\_eval2, me\_instr\_eval1)}
\end{Highlighting}
\end{Shaded}

\begin{verbatim}
## refitting model(s) with ML (instead of REML)
\end{verbatim}

\begin{verbatim}
## Data: InstEval
## Models:
## me_instr_eval2: y ~ service + dept + d + d:service + (1 | s)
## me_instr_eval1: y ~ service + dept + d + d:service + (1 | studage/s)
##                npar    AIC    BIC  logLik deviance Chisq Df Pr(>Chisq)
## me_instr_eval2 1792 236224 252718 -116320   232640                    
## me_instr_eval1 1793 236226 252729 -116320   232640     0  1          1
\end{verbatim}

In conclusion, nesting students in student age doesn't really help.

When you look at the effect size and polarity of d12 in each of the
models above, it varies wildly. This bothered me at first: shouldn't at
least the direction of the effect be consistent? But then I realized
that the regressions aren't giving us inconsistent information, they are
just saying, if this is your model of how things work, this is the
effect we see.

For example, when we don't include department, d12 cannot be
differentiated from the mean score. However, when we include department
and service in the regression, she appears to have a highly significant,
highly positive effect on her rating. However, when we include
dept:service, she suddenly has a highly \emph{negative} effect on
ratings. That is to say that her mix of department and service schedule
explains her high ratings.

By which measure should she be evaluated? Which one is the \emph{truth}?
The answer is that none of them are true, they are simply metrics that
fall out of how you choose to explain lecture ratings.

\hypertarget{mixed-effect-model-data-simulation}{%
\subsection{Mixed Effect Model Data
Simulation}\label{mixed-effect-model-data-simulation}}

Our previous models were a little too complicated for data simulation so
we will train a simple mixed effect model:

\begin{Shaded}
\begin{Highlighting}[]
\NormalTok{simple\_me\_reg }\OtherTok{\textless{}{-}} \FunctionTok{lmer}\NormalTok{(y }\SpecialCharTok{\textasciitilde{}}\NormalTok{ service }\SpecialCharTok{+}\NormalTok{ d }\SpecialCharTok{+}\NormalTok{ (}\DecValTok{1} \SpecialCharTok{|}\NormalTok{ dept), }\AttributeTok{data=}\NormalTok{InstEval)}
\FunctionTok{summary}\NormalTok{(simple\_me\_reg)}
\end{Highlighting}
\end{Shaded}

\begin{verbatim}
## Linear mixed model fit by REML ['lmerMod']
## Formula: y ~ service + d + (1 | dept)
##    Data: InstEval
## 
## REML criterion at convergence: 238259.8
## 
## Scaled residuals: 
##      Min       1Q   Median       3Q      Max 
## -2.80774 -0.75272  0.04866  0.79446  2.69938 
## 
## Random effects:
##  Groups   Name        Variance Std.Dev.
##  dept     (Intercept) 0.01162  0.1078  
##  Residual             1.49308  1.2219  
## Number of obs: 73421, groups:  dept, 14
## 
## Fixed effects:
##               Estimate Std. Error t value
## (Intercept)  3.727e+00  3.839e-01   9.710
## service1    -9.739e-02  1.367e-02  -7.123
## d6          -9.499e-01  4.551e-01  -2.087
## d7           4.307e-01  4.521e-01   0.953
## d8          -1.122e+00  4.293e-01  -2.615
## d12         -7.945e-02  4.570e-01  -0.174
## d13         -2.498e-02  4.441e-01  -0.056
## ...
\end{verbatim}

\begin{verbatim}
## 
## Correlation matrix not shown by default, as p = 1129 > 12.
## Use print(x, correlation=TRUE)  or
##     vcov(x)        if you need it
\end{verbatim}

\end{document}
